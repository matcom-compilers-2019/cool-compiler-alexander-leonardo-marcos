
\documentclass[a4paper, 10pt]{article}
\usepackage[spanish]{babel}
\usepackage[utf8]{inputenc}
\usepackage[top=1 in, left=1 in, right=1 in, bot=1 in]{geometry}
\usepackage{amsmath, amsthm, amsfonts}
\usepackage{graphics}
\usepackage{float}
\usepackage{epsfig}
\usepackage{amssymb}
\usepackage{latexsym}
\usepackage{newlfont}
\usepackage{epstopdf}
\usepackage{amsthm}
\usepackage{epsfig}
\usepackage{caption}
\usepackage{multirow}
\usepackage[colorlinks]{hyperref}
\usepackage[x11names,table]{xcolor}
\usepackage{graphics}
\usepackage{wrapfig}
\usepackage[rflt]{floatflt}
\usepackage{multicol}
\usepackage{longtable,multirow,booktabs}
\usepackage{listings} \lstset {language = Python, basicstyle=\bfseries\ttfamily, keywordstyle = \color{blue}, commentstyle = \bf\color{green}, backgroundcolor = \color{gray!5}, stringstyle = \color{yellow!5}}
\usepackage{titlesec}

% \titleformat{\section}{\filcenter}{}{8pt}{}

\setcounter{secnumdepth}{0}

\title{Simulación de la Cocina de Kojo}
\author{Alexander A. González Fertel C-412\\
		\href{mailto:a.fertel@estudiantes.matcom.uh.cu}{a.fertel@estudiantes.matcom.uh.cu}}
\date{}

\begin{document}
	\maketitle
	% \newpage

	\section{Problema}
	La cocina de Kojo es uno de los puestos de comida rápida en un centro
	comercial. El centro comercial está abierto entre las 10:00 am y las 9:00 pm cada
	día. En este lugar se sirven dos tipos de productos: sándwiches y sushi. Para los
	objetivos de este proyecto se asumirá que existen solo dos tipos de consumidores:
	unos consumen solo sándwiches y los otros consumen solo productos de la gama
	del sushi. En Kojo hay dos períodos de hora pico durante un día de trabajo;
	uno entre las 11:30 am y la 1:30 pm, y el otro entre las 5:00 pm y las 7:00
	pm. El intervalo de tiempo entre el arribo de un consumidor y el de otro no es
	homogéneo pero, por conveniencia, se asumirá que es homogéneo. El intervalo
	de tiempo de los segmentos homogéneos, distribuye de forma exponencial.
	Actualmente dos empleados trabajan todo el día preparando sándwiches y
	sushi para los consumidores. El tiempo de preparación depende del producto en
	cuestión. Estos distribuyen de forma uniforme, en un rango de 3 a 5 minutos 
	para la preparación de sándwiches y entre 5 y 8 minutos para la preparación de
	sushi.
	El administrador de Kojo está muy feliz con el negocio, pero ha estado 
	recibiendo quejas de los consumidores por la demora de sus peticiones. Él está 
	interesado en explorar algunas opciones de distribución del personal para reducir
	el número de quejas. Su interés está centrado en comparar la situación actual con
	una opción alternativa donde se emplea un tercer empleado durante los períodos
	más ocupados. La medida del desempeño de estas opciones estará dada por el
	porciento de consumidores que espera más de 5 minutos por un servicio durante
	el curso de un día de trabajo.
	Se desea obtener el porciento de consumidores que esperan más de 5 minutos
	cuando solo dos empleados están trabajando y este mismo dato agregando un
	empleado en las horas pico.
	
	\section{Modelo}
	Para dicha situación es utilizado un modelo de servidores en paralelo como se muestra en la 
	en la Figura~\ref{fig:1}. Se realizan 2 tipos de simulaciones, {\itshape normales}
	y {\itshape mejoradas}. Las primeras son aquellas en las que tenemos solo 2 empleados
	durante el día, incluyendo los horarios picos. Las que restan son simulaciones en las
	que se añade un tercer empleado a la cocina en los horarios picos.

	En nuestro sistema, es necesario conocer con qué frecuencia llegan consumidores a la cocina,
	lo cual está representado por una {\itshape variable aleatoria exponencial}. Dicha variable aleatoria
	es descrita por su {\itshape valor esperado}, el cual es un parámetro ajustable en la simulación
	y donde dado un valor fijo de este parámetro obtenemos resultados diferentes. La generación de
	variables aleatorias exponenciales se realiza utilizando el método de la {\itshape inversa},
	resultando en: $X = -clogU$, donde $X$ es la variable aleatoria exponencial, $U$ es 
	una variable aleatoria uniforme $U(0, 1)$ y $c$ es la media de la exponencial, es decir,
	si la v.a. exponencial tiene como parámetro $\lambda$, entonces $c = \frac{1}{\lambda}$.
	En la Figura~\ref{fig:2} se pueden ver representadas v.a. exponenciales generadas en este proyecto,
	dados valores de $\lambda$ distintos.

	\begin{figure}[h]
		\begin{center}
			\includegraphics[width=0.6\textwidth]{images/servidores_paralelos.jpg}			
		\end{center}
		\caption{Servidores en paralelo.}
		\label{fig:1}	
	\end{figure}	

	\begin{figure}[h]
		\begin{minipage}{.5\textwidth}
			\centering
			\includegraphics[width=.75\linewidth]{images/exponential.png}			
		\end{minipage}
		\begin{minipage}{.5\textwidth}
			\centering
			\includegraphics[width=.75\linewidth]{images/exponential1.png}			
		\end{minipage}		
		\caption{Funciones de distribución de variables aleatorias exponenciales.}
		\label{fig:2}	
	\end{figure}

	Otro parámetro que cambia en gran medida el resultado de la simulación es qué tipo de producto
	quiere consumir cada cliente, es decir, asumamos que 1 de cada 2 clientes desea consumir
	un sándwich, dado esto, se puede afirmar sin simular la situación que aproximadamente
	al menos la mitad de los clientes van a pasar más de 5 minutos esperando, dado que preparar
	sushi demora entre 5 y 8 minutos. Por tanto, la proporción de consumidores que quieren cada
	producto se comporta como una v.a. {\bfseries Bernoulli(p)}.

	Además, durante las horas pico los consumidores llegan siguiendo una distribución exponencial que
	lógicamente tiene un parámetro mayor que la usada fuera de estas horas. Dicho parámetro también es ajustable
	en la simulación, es la variable \verb!PH_MEAN! en el archivo \verb!config.py!.

	Cada empleado de la cocina es un servidor que atiende cada pedido dado el producto 
	que se desee indistintamente. Un consumidor arriba a la cocina y se coloca en la cola,
	presentándose las siguientes situaciones:

	\begin{itemize}
		\item
			Existe al menos un empleado libre, por lo que el consumidor puede ser inmediatamente atendido y pasa
			a ocupar el primer servidor vacío.
		\item
			Dado que no existen servidores vacíos, el cliente se encola hasta nuevo aviso. 
	\end{itemize}

	En el modelo, el tiempo está representado en segundos, es decir, todas las variables del sistema que 
	tienen relación con el tiempo son números equivalentes a segundos. Por ejemplo, el resultado de una simulación
	para una frecuencia de arribo de 5 minutos se obtendría luego de cambiar en el archivo \verb!config.py! la variable
	\verb!MEAN! a 300. Al llegar el tiempo de cierre de la tienda en el modelo, se deja de generar arribos y se termina
	de atender a los clientes.

	\section{Discusión de los Resultados}
	Se realizaron simulaciones para disímiles valores de los parámetros, la cantidad de simulaciones diferentes es
	la variable \verb!RUNS!, que al momento de escribir este informe, tenía el valor 42674. Toda la información
	considerada importante de una simulación se persiste en el directorio \verb!logs!. En el directorio \verb!results!
	se persisten archivos \verb!csv! con pares de porcientos para cada simulación, donde cada par significa cuántos
	clientes estuvieron más de 5 minutos en el sistema, la primera columna siendo una simulación normal y la segunda
	una simulación con 3 empleados.

	Para cada valor múltiplo de 30 entre 1 y 10 minutos se realizaron simulaciones donde este representaba el valor
	esperado de la v.a. exponencial que dictaba la frecuencia de arribos al sistema. Para cada valor de este parámetro se
	realizaron simulaciones con múltiplos de 0.05 entre 0 y 1 como valor del parámetro que representaba la proporción de
	consumidores de sándwiches que llegaban al sistema, es decir, si el valor de este parámetro es 0.3, entonces aproximadamente
	30 de cada 100 personas que llegaran al sistema, consumen sándwiches y el resto sushi.

	Se implementó un mecanismo para conocer qué tanto mejora {\bfseries en promedio} poner un tercer empleado con respecto a
	una cocina con dos. La Figura~\ref{fig:3} muestra cómo se comporta la simulación con respecto a los parámetros de las
	v.a. exponenciales y la proporción de personas consumidoras de un producto u otro. En ella se puede ver que a medida
	que hay menos clientes que consumen sushi, aumenta significativamente la calidad del sistema, dado que si una persona
	consume sushi está condicionada a esperar entre 5 y 8 minutos. También se puede ver una mejora al disminuir la frecuencia
	con que llegan personas a la cocina (aumentar el valor esperado), por razones obvias.

	\begin{figure}[h]
		\centering
		\includegraphics[width=.6\textwidth]{images/behaviour.png}			
		\caption{Comportamiento de los parámetros.}
		\label{fig:3}	
	\end{figure}

	Le recomendaríamos al dueño de la cocina añadir a un tercer empleado, puesto que para parámetros razonables de la simulación,
	representa una mejora considerable al negocio.

\end{document}
